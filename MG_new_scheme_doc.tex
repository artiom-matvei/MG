\documentclass[]{article}
\usepackage{amsmath}
\usepackage{amsfonts}
\usepackage{tikz}
%opening
\title{}
\author{}

\begin{document}

\maketitle

%\begin{abstract}
%\end{abstract}

%\section{}
Given the following elliptic problem (1) with Neumann boundary conditions (2) and holding the Integral compatibility condition (3)
\begin{equation}
\nabla \cdot(H \nabla \phi) = \nabla \cdot F \textmd{ on } \Omega
\end{equation}
\begin{equation}  \nabla \phi \cdot \vec{n} = 0 \textmd{ on } \partial \Omega 
\end{equation}
\begin{equation} F \cdot \vec{n} = 0 \textmd{ on } \partial \Omega 
\end{equation}
\begin{equation*}
\Omega= [0,0] \times [\pi,\pi]
\end{equation*}
Let  \[ \nabla \cdot F = divF\].

Then the approximate version of the above is
\begin{multline*}
\nabla \cdot(H \nabla \phi) (x,y) \approx \frac 1 {h^2} \Bigg\lbrace 
\phi (x+h,y)H\left(x+\frac h 2,y\right)+\phi(x-h,y)H\left(x-\frac h 2,y\right) \\
+\phi(x,y+h)H\left(x,y+\frac h 2\right)+\phi(x,y-h)H\left(x,y-\frac h2\right)\\
-\phi(x,y)\left[H\left(x+\frac h2,y\right)+H\left(x- \frac h2,y\right)+ H\left(x,y+\frac h2\right)+H\left(x,y- \frac h2 \right)\right]
\Bigg\rbrace
\end{multline*}
and
\begin{multline}
\nabla \cdot F(x,y)=divF(x,y) \approx \frac 1 h \Bigg\lbrace F\left(x+\frac h 2,y\right) -F\left(x-\frac h 2,y\right)\\ +F\left(x,y+\frac h 2\right) - F\left(x,y+\frac h 2\right)\Bigg\rbrace
\end{multline}
Then by discretizing the space $\Omega$ with n=4, we have the grid below.


\begin{tikzpicture}
\draw[step=2cm,gray,dashed] (0,0) grid (8,8);
\draw (0,0) node[fill=white, outer sep=5pt] {$0$};
\draw (8,0) node[fill=white, outer sep=5pt] {$(\pi,0)$};
\draw (0,8) node[fill=white, outer sep=5pt] {$(0,\pi)$};
\draw (1,-1) node{$\phi$};
\draw (-1,1) node{$\phi$};
\draw (-1,-1) node{$\phi$};
\draw (1,1) node{$\phi_{1,1}$};
\draw (3,3) node{$\phi$};
\draw (1,3) node{$\phi_{1,2}$};
\draw (3,1) node{$\phi_{2,1}$};
\draw (-1,3) node{$\phi$};
\draw (3,-1) node{$\phi$};
\draw (-1,-3) node{$0$};
\draw (1,-3) node{$1$};
\draw (3,-3) node{$2$};
\draw (5,-3) node{$3$};
\draw (7,-3) node{$4$};
\draw (9,-3) node{$5$};
\draw (-3,-1) node{$0$};
\draw (-3,1) node{$1$};
\draw (-3,3) node{$2$};
\draw (-3,5) node{$3$};
\draw (-3,7) node{$4$};
\draw (-3,9) node{$5$};
\draw (1,0) node[fill=white, outer sep=5 pt]  {$H_{1,1,1}$};
\draw (1,2) node[fill=white, outer sep=5 pt]  {$H_{1,2,1}$};
\draw (0,1) node[fill=white, outer sep=5 pt]  {$H_{1,1,0}$};
\draw (2,1) node[fill=white, outer sep=5 pt]  {$H_{2,1,0}$}; \end{tikzpicture}


$\Omega$ is the area inside the dashed grid.
Note that $\phi$ and $divF$ are situated in the same place and indexed in the same way.
The same is true for $H$ and $F$.

Thus (1) is equivalent to (5)
\begin{multline}
\frac 1 {h^2} \Big[\phi_{i+1,j}H_{i+1,j,0}+\phi_{i-1,j}H_{i,j,0}+\phi_{i,j+1}H_{i,j+1,1}+\phi_{i,j-1}H_{i,j,1}\\ -\phi_{i,j}\big(H_{i+1,j,0}+H_{i,j,0}+H_{i,j+1,1}+H_{i,j,1}\big) \Big]=divF_{i,j}
\end{multline}
where
\begin{equation*}
divF_{i,j}=\frac 1 h \big[F_{i+1,j,0}-F_{i,j,0}+F_{i,j+1,1}-F_{i,j,1}\big]
\end{equation*}
Equation (2) is equivalent to the 4 equations below
\begin{align*}
\phi_{i,0}&=\phi_{i,1}\\
\phi_{i,n+1}&=\phi_{i,n}\\
\phi_{0,j}&=\phi_{1,j}\\
\phi_{n+1,j}&=\phi_{n,j}
\end{align*}
And equation (3) becomes simply
\begin{align*}
F_{1,j,0}&=0\\
F_{n+1,j,0}&=0\\
F_{i,1,1}&=0\\
F_{i,n+1,1}&=0
\end{align*}
\end{document}
